%\specsection{ОПРЕДЕЛЕНИЯ}
\section*{\centering ОПРЕДЕЛЕНИЯ}
\addcontentsline{toc}{section}{ОПРЕДЕЛЕНИЯ}

%Application Programming Interface --- «Интерфейс прикладного программирования» - набор инструментов программирования, который разрешает программе взаимодействовать с другой программой или операционной системой и помогает разработчикам программного обеспечения создавать свои собственные приложения~\cite{api}.

%Протокол --- формализованные правила, определяющие последовательность и формат сообщений, которыми обмениваются сетевые компоненты, лежащие на одном уровне модели сетевого взаимодействия в разных узлах.~\cite{tcpip_lora}.

%IP-пакет или пакет --- отформатированная единица данных, переносимая сетью с коммутацией пакетов.~\cite{tcpip_lora}.

База данных --- именованная часть информационного хранилища, структура которой описывается не языке некоторого модели данных \cite{definebase}.

Система управления базами данных --- совокупность программ и языковых средств, предназначенных для управления данными в базе данных, ведения базы данных и обеспечения взаимодействия её с прикладными программами.
