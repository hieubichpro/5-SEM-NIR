%\specsection{ЗАКЛЮЧЕНИЕ}
\section*{\centering ЗАКЛЮЧЕНИЕ}
\addcontentsline{toc}{section}{ЗАКЛЮЧЕНИЕ}

Из результатов сравнения было выявлено, что каждая из архитектур СУБД различена относительно других архитектур по своим преимуществам и недостаткам. Таким образом, выбор архитектуры СУБД зависит в первую очередь от типа проекта, которые необходимо проектировать, и нет единой архитектуры, которая была бы наилучшей для решения всех задач. Разработанная в ходе работы методика позволяет сравнивать архитектуры СУБД по модели данных, по способу доступа к базе данных и определять, какая из архитектур подходит для решения конкретной задачи.

Цель данной работы достигнута. В ходе данной работы были решены все задачи:
\begin{itemize}
	\item проведен анализ предметной области системы управления базами данных;
	\item проведен обзор существующих систем управления базами данных по модели данных и по способу доступа к базе данных;
	\item сформулированы критерии сравнения этих систем;
	\item классифицированны существующие системы управления базами данных по модели данных и по способу доступа к базе данных.
\end{itemize}


