\section*{\centering РЕФЕРАТ}
\addcontentsline{toc}{section}{РЕФЕРАТ}
\setcounter{page}{2}

Научно-исследовательская работа \pageref{LastPage} с., \totalfigures\ рис., \totaltables\ табл., 10 ист., 1 прил.

БАЗЫ ДАННЫХ, СИСТЕМЫ УПРАВЛЕНИЯ БАЗАМИ ДАННЫХ, ФАЙЛ-СЕРВЕРНЫЕ, КЛИЕНТ-СЕРВЕРНЫЕ, ВСТРАИВАЕМЫЕ, ИЕРАРХИЧЕСКИЕ СУБД, СЕТЕВЫЕ СУБД, РЕЛЯЦИОННАЯ СУБД

Объект исследования --- система управления базами данных.

Целью работы является проведение анализа существующих систем управления базами данных по модели данных и по способу доступа к базе данных.
Поставленная цель достигается путем рассмотрения и классификации существующих систем управления базами данных.

Результат данной работы показал, что каждая система управления базами данных имеет ряд преимуществ и недостаткам и применим в зависимости от требований и ограничений к задаче.