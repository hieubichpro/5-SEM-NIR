%\specsection{ВВЕДЕНИЕ}
\section*{\centering ВВЕДЕНИЕ}
\addcontentsline{toc}{section}{ВВЕДЕНИЕ}

В современном информационном мире, где объемы данных постоянно растут, системы управления базами данных играют ключевую роль в обеспечении эффективного хранения, организации и управления информацией.
Эти мощные программные инструменты стали неотъемлемой частью информационной инфраструктуры предприятий, образовательных учреждений, государственных учреждений и многих других организаций.

Целью работы является проведение анализа существующих систем управления базами данных по модели данных и по способу доступа к базе данных.

Для достижения поставленной цели необходимо решить следующие задачи:
\begin{itemize}[label=---]
	\item провести анализ предметной области системы управления базами данных;
	\item провести обзор существующих систем управления базами данных по модели данных и по способу доступа к базе данных;
	\item сформулировать критерии сравнения этих систем;
	\item классифицировать существующие системы управления базами данных по модели данных и по способу доступа к базе данных.
\end{itemize}