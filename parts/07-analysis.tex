\section{Анализ предметной области}
Прежде чем приступить к теме исследования, следует описать предмет исследования и связанные с ним термины.
В данной секции описаны понятия, которые будут использоваться в работе.

\subsection{Базовые понятия и термины}

Данные --- это информация, зафиксированная в некоторой форме, пригодной для последующей обработки, передачи и хранения, например, находящаяся в памяти ЭВМ или подготовленная для ввода в ЭВМ \cite{basedef}.

База данных --- структурированное поименованное хранилище информации \cite{basedef}.

СУБД --- специализированное программное обеспечение, обеспечивающее доступ к базе данных как к совокупности её структурных единиц.
СУБД необходима для создания и поддержки базы данных информационной системы в той же степени, как для разработки программы на алгоритмическом языке транслятор \cite{basedef}.

Программные составляющие СУБД включают в себя ядро и сервисные средства (утилиты) \cite{basedef}.
\begin{enumerate}
	\item Ядро СУБД --- это набор программных модулей, необходимый и достаточный для создания и поддержания БД, то есть универсальная часть, решающая стандартные задачи по информационному обслуживанию пользователей.
	\item Сервисные программы предоставляют пользователям ряд дополнительных возможностей и услуг, зависящих от описываемой предметной области и потребностей конкретного пользователя.
\end{enumerate}
\clearpage
Модель данных --- это совокупность правил порождения структур данных в базе данных, операций над ними, а также ограничений целостности, определяющих допустимые связи и значения данных, последовательность их изменения.

\subsection{Способы классификации СУБД}
Классификация СУБД является важным этапом в области баз данных, поскольку позволяет систематизировать и структурировать разнообразие существующих решений.

Эта задача включает определение различных критериев, которые позволяют группировать СУБД по их основным характеристикам.
\begin{enumerate}
	\item По модели данных: означает разделение СУБД на группы в соответствии с тем, как они моделируют и организуют данные внутри базы данных.
	\item По способу доступ к БД: предполагает разделение их в зависимости от того, каким образом приложения и пользователи могут взаимодействовать с данными в базе.
\end{enumerate}